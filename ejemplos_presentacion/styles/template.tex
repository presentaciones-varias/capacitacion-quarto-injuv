
\usepackage{tikz}
\usepackage{graphicx}
\usepackage{etoolbox}

% Redefinir el estilo del título de cada frame
\setbeamertemplate{frametitle}{%
  \nointerlineskip
  \begin{beamercolorbox}[wd=\paperwidth,ht=0cm,dp=0cm]{frametitle}
    \begin{tikzpicture}[remember picture,overlay]
      % Imagen de fondo como barra (ajusta la ruta y tamaño)
      \node[anchor=north west, inner sep=0] at (0,0) {\includegraphics[width=\paperwidth,height=2cm]{images/fondo2.png}};
      % Texto del título encima de la imagen
      \node[anchor=west, xshift=0.2cm, yshift=-0.5cm, text=white, font=\large\bfseries] at (0,0) {\insertframetitle};
    \end{tikzpicture}
  \end{beamercolorbox}
  \vspace*{0.5cm} 
}


\newcommand{\CodeSize}{\tiny} % clase para modificar el tamaño del codigo
% puede ser \tiny, \footnotesize, \huge, etc.

\AtBeginEnvironment{Highlighting}{\CodeSize} % Cada vez que comience un bloque Highlighting, aplica el tamaño definido

\AtBeginEnvironment{verbatim}{\CodeSize}     % Cuando R imprime una tabla, un error o un texto plano, eso no va en Highlighting sino en verbatim

\makeatletter
\@ifpackageloaded{listings}{                % Si el template cargara el paquete listings, se usa lstlisting en vez de Highlighting
  \lstset{basicstyle=\ttfamily\CodeSize, breaklines=true, columns=fullflexible, keepspaces=true}
}{}

\@ifpackageloaded{minted}{                  % otro backend para resaltar código
  \setminted{fontsize=\CodeSize, breaklines, autogobble}%
}{}
\makeatother
